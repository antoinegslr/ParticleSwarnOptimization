\documentclass[11pt]{article}

% Language setting
% Replace `english' with e.g. `spanish' to change the document language
\usepackage[english]{babel}

% Set page size and margins
% Replace `letterpaper' with `a4paper' for UK/EU standard size
\usepackage[a4paper,top=1.75cm,bottom=1.75cm,left=1.5cm,right=1.5cm,marginparwidth=1.75cm]{geometry}

% Useful packages
\usepackage{amsmath,amssymb,amsfonts,multicol}
\usepackage{graphicx}
\usepackage{caption}
\usepackage[colorlinks=true, allcolors=blue]{hyperref}
\newenvironment{Figure}
  {\par\medskip\noindent\minipage{\linewidth}}
  {\endminipage\par\medskip}

\title{\textsf{\textbf{f}}}
\author{Antoine GISSLER}
\date{January 10th, 2023}

\begin{document}
\noindent\huge\textbf{\textsf{Generation of optimized structures using Particle Swarn Optimization (PSO)}}\normalsize\vspace{1em}\\
\large \textsf{Antoine GISSLER -- Sorbonne Université\\January 2023}\vspace{0.5em}\\
\begin{tabular}{p{18cm}}
    \hline
    \vspace{-0.25em}\begin{minipage}{\textwidth}
        \small\textbf{Note:} This bibliographic report focuses on the article "Novel phases in ammonia-water mixtures under pressure" published by \textsc{Naden Robinson} et al. \cite{original}, which used Particle Swarn Optimization for the determination of the phase diagram of ammonia-water mixtures for high pressures and temperatures. It also uses two other articles from \textsc{Wang} et al. \cite{PhysRevB.82.094116,WANG20122063}, explaining the principles of Particle Swarn Optimization.\normalsize
    \end{minipage}\vspace{0.5em}
    \\
    \hline
\end{tabular}
\begin{multicols}{2}
\section*{Introduction}

The nanoscopic scale of matter makes it possible to understand the phases of matter and its transitions when subject to changing conditions, such as pressure or temperature. Indeed, different experimental conditions can lead to different states. Moreover, these changes can lead to variation of properties of materials, which could lead to serious issues in fields where they have expected to withstand critical infrastructure (aeronautics, nuclear powerplants, etc...). Yet, a full understanding of every material is hard to achieve: even for common elements like water, there is still today space for discussion over some of its solid phases \cite{Hansen2021-bk}.

\noindent Up to recently, crystalline structure for materials was obtained through experimental studies: X-Ray diffraction (XRD) being almost the norm in order to characterize anything in Material Science. However, this implies that the crystallized experimental structure shall be accessible, which requires consequent setups in the case of extreme conditions being studied. In the case of a complete theoretical study through numerical calculations, it implies to find the global minimum of the potential energy surface of the molecule, which depends on many parameters (for a molecule containing $N$ atoms, it can go up to $3N-3$ degrees of freedom, including bond length, angles of rotation and torsion). Yet, as simple as the concept may seem, it hides a very complex truth: finding the global minimum of a ensemble containing many parameters, is not an easy task at all. Without any hint on initial configurations to begin optimization with, simulations can get stuck in local minima. Moreover, finding this minimum in the case of a crystalline phase is also conditionned to the right selection for the symmetry of the crystal: yet again increasing the complexity.

\noindent The quest for finding global minima in the case of molecular systems has been done through various approaches (visual presentation in Figure 1):
\begin{itemize}
    \itemsep0em
    \item[\textbf{(a)}] \textbf{Monte Carlo:} \small involves random changes on parameters and accept the modification using probabilistic techniques \cite{PhysRevB.95.144104}\normalsize
    \item[\textbf{(b)}] \textbf{Simulated annealing:} \small the temperature of the system is increased (so that every configuration is accessible), and then slowly decreased for the system to converge to global minimum in a Monte-Carlo fashion \cite{B003447I}\normalsize
    \item[\textbf{(c)}] \textbf{Minima hopping:} \small perturbates the configuration at local minimum to explore nearbies ; if a new configuration has lower energy, the configuration is updated ; if not the perturbation module is increased \cite{minima_hopping}\normalsize
    \item[\textbf{(c')}] \textbf{Basin hopping:} \small similar to minima hopping, but focuses rather on the basins of potential energy\normalsize
    \item[\textbf{(d)}] \textbf{Metadynamics:} \small enhances the sampling of rare events, and thus permits a quasi-total sampling of the potential energy surface (PES) ; in this case the global minimum can be easily found \cite{C2CE06642D}\normalsize
    \item[\textbf{(f)}] \textbf{Genetic algorithm:} \small inspired by the principles of evolution of living species, theses algorithm induce selection, mutation, and recombination to optimized configurations \cite{Falls2020}\normalsize
\end{itemize}\normalsize
\medskip

              \noindent \includegraphics[width=\columnwidth]{figures/optim_figures.png}
                \captionof{figure}{Schematic explanation of various crystal structure prediction methods, by Falls et al. \cite{Falls2020}}\medskip
\newpage
\normalsize Each method has its own advantages and inconvenients, and has been probed for the generation of configurations. Thus, there is still no consensus for a specific method, as techniques can be better in some cases. \vspace{1em}

This report, based on the publication from Naden Robinson et al. \cite{original}, tackles the application of another generalized optimization algorithm (Particle Swarn Optimization) for the purposes of potential energy surfaces exploration and the determination of the optimized geometry in systems when applied extreme conditions. The study is based on the exploration of four different ammonia-water mixtures present inside planets of our solar system (especially Uranus and Neptune). In these mantles, there exist some particularly harsh conditions, that are not existing in our planet: indeed, high temperatures and pressure can be reached (here, pressure is sampled from 1 to 1000 GPa); thus accessing zones of the phase diagram of this mixture that were never explored (theoretically nor experimentally) previously: making it impossible to have first guesses or initial configurations, which are necessary with conventional optimization techniques. As a matter of consequence, using a smart optimization technique is, of course, of high importance.

\section*{Particle Swarn Optimization}
\subsection*{Theoretical background}
Particle Swarm Optimization (also called PSO) is a population-based optimization algorithm that simulates the social behavior of birds or insects, such as flocking or swarming. Every iteration, one optimization simulation's (called a particle) trajectory is inspired by its own personal local minimum (\verb+pbest+) that it can reach through gradient descent, and the global (on all particle) minimum (\verb+gbest+). Its application for structure discovery was introduced by Wang et al. \cite{PhysRevB.82.094116}, and transcribed into a software (called \textsc{calypso}) by the same authors \cite{WANG20122063}. The particle swarn optimization presents a simple algorithm:
\begin{enumerate}
\itemsep0em
    \item $t=0$ -- Initial generation of random structures using symmetry restraints
    \item Local optimization of every structure
    \item Exclusion of similar structures (through bond characterization matrix)
    \item Generation of new structures by PSO, using personal and flock's histories (see below)
    \item Repeating steps 2, 3 and 4 until convergence is reached
    \item Returns the configuration with the lowest energy
\end{enumerate}
At first, structures are randomly (with a uniform distribution) generated with a symmetry constraint within the 230 space groups. And once a particular group is selected, the algorithm makes sure not to generate structures that would be identical with a symmetry operation.

The calculation of new structures by PSO formulas is carried out for each parameter individually.
For the $i^\text{th}$ particle, we have for the calculation of the updated coordinates for dimension $j$:
\begin{equation}
    x_{i,j}^{t+1}=x_{i,j}^{t}+v_{i,j}^{t+1}
\end{equation}
With $v_{i,j}^{t+1}$ its velocity for the $j^\text{th}$ dimension:
\begin{equation}
    v_{i,j}^{t+1}=\omega v_{i,j}^t+c_1r_1(\verb+pbest+_{i,j}^t-x_{i,j}^t)+c_2r_2(\verb+gbest+_{i,j}^t-x_{i,j}^t)
\end{equation}
With $\omega$ being the inertia weight (translating the importance or not of previous velocity), $c_1$ and $c_2$ being respectively the self-confidence and the swarm confidence factor, and $r_i$ being random parameters. By changing the fixed parameters (by fixing them at first, or making them evoluate throughout the simulaton), it is possible to go from global jump, with long hoppings, to a precise localization of the minimum in its basin.
A schematic of an iteration of generation of new coordinates for a particle on a specific coordinate is shown in Figure 2. 
\bigskip

              \noindent \includegraphics[width=\columnwidth]{figures/PSO.png}
                \captionof{figure}{Schematic explanation of particle swarn optimization on one parameter, by Wang et al. \cite{PhysRevB.82.094116}}\medskip
The advantage of such method is thus clearly visible: by taking into account samplings at various zones of the PES, and forcing particles to travel to a zone of interest, it is possible to force particles to get out of their local minima from one hand, but also to provide a better sampling around zones of interest (some will go further, some before the global minimum found at an iteration, which may lead to the discovery of a better minimum, etc...).\vspace{1em}

Authors announce very interesting results in both their introduction article and the presentation of \textsc{calypso} \cite{PhysRevB.82.094116,WANG20122063}: starting from scratch, and using either DFT or empirical potentials (using the GULP code), it was possible to obtain optimized structures with less than 10 PSO generations (representing around 300 structures).
\subsection*{Programming PSO, trial over a simple two-dimensional study case}
The PSO algorithm was reproduced on Python for this bibliographic report using the equations that were stated previously, in order to understand more its principle. The code that was developped following the above instructions, is available on Github (\href{https://github.com/antoinegslr/ParticleSwarnOptimization}{link here}). Our algorithm was applied on a similar case study (yet with a lower dimensionality for the sake of simplicity): the Eggholder trial function.
\begin{multline*}
    f(x,y)=-(y+47)\sin\left(\sqrt{\left|\frac{x}{2}+(y+47)\right|}\right)\\-x\sin\left(\sqrt{\left|\frac{x}{2}-(y+47)\right|}\right)
\end{multline*}
As it would be in the case of molecular systems, the function surface is full of hills and wells, with very steep walls in between them. In the case of conventional sampling methods, it would be either very long to sample (hitting only a few units of $x$ away from a minimum gives very different results), or quite easy to get stuck in local minima.\\
In this example, we randomly placed particles (following a uniform distribution) in an initial area: $(x,y)\in\left[-512,512\right]^2$ and were affected random velocities along each dimension ($|v_i|<50$). In order to converge quickly, $\omega$ was set at 0.6 at the beginning, and linearly decreased to 0.3 until the end of the loop; $c_1$ and $c_2$ were equally defined to 1 (equal confidence).\\
As told previously, the particles quickly adopt a similar behavior to a flock of birds, which is observable in Figure 3.
\bigskip

\noindent\includegraphics[width=0.5\columnwidth]{figures/ite0.png}\includegraphics[width=0.5\columnwidth]{figures/ite1.png}
\captionof{figure}{Visualization of particles at initial generation (left) and after one iteration (right)}\medskip
Using only one PSO iteration, we observed the quick movement of particles towards the \verb+gbest+ area, which allowed some further discoveries, as visible in Figure 4.
\bigskip

\noindent\begin{center}
    \includegraphics[width=0.9\columnwidth]{figures/eggholder.png}
\end{center} \captionof{figure}{Displacement of global best minimum in the first iterations using the Eggholder function}\medskip 
In this example, the global best minimum managed to move from a basin to another without any problem, even though the spatial steps that separate every minimum are important. It becomes also clear that the more particles are considered, the easier it will be to find new global best minima, although it comes at the cost of more computational time: a compromise needs to be found to allow sufficient sampling.

\section*{Obtained results and comparison to other methods}
Indeed, this method surely seems very advantageous to others when facing unknown crystal phases determination: in the considered planets (Uranus and Neptune), layers of the inner mantle are mainly composed of methane, ammonium and ice water, and are subject to pressure going from 0.1 to 2.5 Mbar \cite{https://doi.org/10.1029/JB085iB01p00225}. It is predicted that the presence of both elements is almost equivalent, letting their stoichiometry unknown (it could be 1:1, or one element could be more used than another). However, as ammonia and water can form hydrogen-bonded networks when they are brought closer in some circumstances, it is possible to see the formation of new phases, lowered in energy. It is thus expected that the evolution of pressure might let new phases appear at high pressure. Experimental studies were conducted at a few hundreds of GPa previously, and already gave some hints about the rearrangements in place \cite{experimental}, but they do not tell the whole story.\\
In this case, only mixtures of ammonia and water were studied, and multiple stoichiometric factors were tested, which are all supposedly possible to be observed in those planet's layers. At first, 16 formula units of $( \mathrm{H}_2\mathrm{O})_{X}(\mathrm{N}\mathrm{H}_3)_{Y}$ were tried, however it was soon reduced to only four stable mixtures: ammonia monohydrate (AMH), ammonia dihydrate (ADH), ammonia hemihydrate (AHH), and ammonia quarterhydrate (AQH). The three first configurations are present on our planet at low pressures, making them easy options to consider for stable configurations. However, the last mixture turned out to present some stable crystalline phases at high pressures.\vspace{1em}

Particle Swarn Optimization that applied on the already-known ammonia hydrates still allowed to make some new discoveries: some phases that were not present in some previously determined phase diagram are actually valid candidates for some pressures: it is the case, for instance, for AMH, with both $P4_3$ and $P2_1/m$ structures that did not appear in the litterature: using PSO in those high-pressure phase evolution studies thus allowed to discover a better stability of the crystal, before changing to the ionic phase $(\mathrm{OH}^-)(\mathrm{NH}_4^+)$. In previous studies, such as the one conducted by Bethkenhagen and others which was using a genetic evolution based algorithm (XtalOpt) \cite{doi:10.1021/acs.jpca.5b07854}, concluded first that all phases of AMH would be decomposed into ionic phases below 120 GPa. Thus, this shows that the latter method is less accurate that PSO, as some phases were forgotten. A richer phase diagram was also found for ADH, with phases being stable up to around 100 GPa. These new phases are expected due to the facilitated proton transfer that can occur in the system with the high pressure (in all mixtures): the energy of the system decreases, makes it possible to consider these phases. This phenomenon also exists for AHH, and is shown in Figure 5.\bigskip

\noindent\begin{center}
    \includegraphics[width=0.9\columnwidth]{figures/hbond.png}
\end{center}
\captionof{figure}{Structural relation between AHH-II at 10 GPa and $P\bar{3}m1$ at 150 GPa, by Naden Robinson et al. \cite{original}}\medskip 

This figure highlights the presence of unusual structural motifs, here the ammonium oxide $\mathrm{O}^{2-}(\mathrm{N}\mathrm{H}_4^+)_2$.\\
Moreover, using PSO also made it possible to discover the stability of another ammonia-water mixture: ammonia quarterhydrate (AQH). This phases begins to exist at already high pressures (starting from 8.5 GPa), and decomposes back into ionic phases again around 300 GPa.

\bigskip
\noindent\begin{center}
    \includegraphics[width=0.9\columnwidth]{figures/AQH.png}
\end{center}
\captionof{figure}{The AQH-$P2_1/m$ structure at 40 GPa, represented from Naden Robinson et al. data \cite{original}}\medskip 

The AQH structure also highlights the presence of another unusual structural motif: $\mathrm{O}^{2-}(\mathrm{N}\mathrm{H}_4^+)_2(\mathrm{N}\mathrm{H}_3)_2$ (also written $\mathrm{O}^{2-}(\mathrm{N}_2\mathrm{H}_7^+)_2$).\vspace{1em}

\noindent In the end, it was also shown that no stoichiometry other than AMH, ADH, AHH or AQH would ever be stable, by using a convex hull diagram (which consists in representing the difference of enthalpy of formation between the considered ammonia-water crystal and the seperated constituants). Indeed, other proportions systematically lead to the dissociation into the previously listed phases, or the basic constituents. This can be observed by the fact that, for other proportions only, structures which were predicted were never able to get a negative difference, making it unstable. Thus, this study leads to the final phase diagram, both at $T=0K$ and $T=300K$ (by taking into account phonon dispersion): this diagram in shown in Figure 7.

\section*{Conclusion}
The sampling of various binary ammonia-water mixtures using Particle Swarn Optimization, performed in the studied article, made it possible to discover new phases, extending the domain of stability that was initially found in previous studies. As a consequence, this sampling technique has showed its advantages in this particular case, as being more rigorous than others, and permitting an efficient sampling (less time is spent in the upper part of the hills of the PES and takes care of symmetry constraints). Yet, it is still important to note that in other cases, this method might not be the most efficient: PSO does not handle dynamics and is less useful in the case of the presence of experimental data (which gives a starting point and entitles for easier sampling). There is no gold standard, one has to get the best compromise for their studies.

\end{multicols}
\begin{figure}[h]
    \centering
    \includegraphics[width=0.9\textwidth]{figures/phase-diagram.png}
    \caption{Phase stability ranges for binary ammonia-water mixtures as a function of pressure, for the ground state (left) and at T = 300 K (right), by Naden Robinson et al. \cite{original}}
\end{figure}
\begin{multicols}{2}
 
\bibliographystyle{bibstyle.bst}
\normalsize\bibliography{articles}\normalsize
\end{multicols}

\end{document}