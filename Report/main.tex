\documentclass[11pt]{article}

% Language setting
% Replace `english' with e.g. `spanish' to change the document language
\usepackage[english]{babel}

% Set page size and margins
% Replace `letterpaper' with `a4paper' for UK/EU standard size
\usepackage[a4paper,top=1.75cm,bottom=1.75cm,left=1.75cm,right=1.75cm,marginparwidth=1.5cm]{geometry}

% Useful packages
\usepackage{amsmath,amssymb,amsfonts,multicol}
\usepackage{graphicx}
\usepackage[colorlinks=true, allcolors=blue]{hyperref}

\title{\textsf{\textbf{f}}}
\author{Antoine GISSLER}
\date{January 10th, 2023}

\begin{document}
\noindent\huge\textbf{\textsf{Generation of optimized structures using Particle Swarn Optimization (PSO)}}\normalsize\vspace{1em}\\
\large \textsf{Antoine GISSLER\\Sorbonne Université}
\begin{multicols}{2}
[
\section{Introduction}
]

For as long as both theoretical and technological progress entitled scientists to do, there has been a wide interest to understand the nanoscopic scale of matter. Indeed, doing so makes it possible to understand the processes of transition and their results when matter is subject to changing conditions, such as pressure or temperature. Indeed, it has been observed that applying new conditions could alter various variables in matter: in the case of water ices, the symmetry and the number of H-bonds that are possible can be altered by the variation of pressure. Moreover, these changes can lead to variation of properties of materials, which could lead to serious issues in fields where they have expected to withstand critical infrastructure (aeronautics, nuclear powerplants, etc...). Yet, a full understanding of every material is hard to achieve: even for very common elements such as water, there is still today space for discussion over some of its solid phases \cite{Hansen2021-bk}. \\
Up to recently, crystalline structure for materials was obtained (or at least helped by it) through experimental studies: X-Ray diffraction (XRD) being almost the norm in order to characterize anything in Material Science. However, this implies that the crystallized experimental structure shall be accessible, which requires consequent setups in the case of extreme conditions being studied. In the case of a complete theoretical study through numerical calculations, it implies to find the global minimum of the potential energy surface of the molecule, which depends of many parameters (for a molecule containing $N$ atoms, it can go up to $3N-3$ degrees of freedom, including bond length, angles of rotation and torsion). Yet, as simple as the concept may seem, it hides a very complex truth: finding the global minimum of a ensemble containing many parameters, is not an easy task at all. Without any hint on initial configurations to begin optimization with, simulations can get stuck in local minima. Moreover, finding this minimum in the case of a crystalline phase is also conditionned to the right selection for the symmetry of the crystal: yet again increasing the complexity.\vspace{1em}

\noindent The quest for finding global minima in the case of molecular systems has been done through various approaches:
\begin{itemize}
    \itemsep0em
    \item \textbf{Monte Carlo:} induce random changes on parameters and accept the modification if energy is lowered or with a certain probability \cite{PhysRevB.95.144104}
    \item \textbf{Simulated annealing:} the temperature of the system is increased so that every configuration is accessible, and then is slowly decreased for the system (helped by probabilities) to converge to the global minimum
    \item \textbf{Minima hopping}
    \item \textbf{Basin hopping}
    \item \textbf{Metadynamics}
    \item \textbf{Genetic algorithm:} induce mutations to optimized configurations to try to cross energy barriers
    \item \textbf{Random Sampling methods}
\end{itemize}
Each method has its own advantages and inconvenients, and has been probed for the generation of configurations. Thus, there is still no consensus for a specific method, as techniques can be better in some cases. \vspace{1em}

This report, based on the publication from Robinson et al. \cite{original}, tackles the application of a generalized optimization algorithm (Particle Swarn Optimization) for the purposes of potential energy surfaces exploration and the determination of the optimized geometry for particular conditions.

%Faire une introduction sur la nécessité de devoir générer des configurations par la théorie dans le cas de conditions expérimentales exotiques (on ne peut pas se baser sur des expériences précédentes car il n'y en a pas lol)
\section{Particle Swarn Optimization}
\subsection{Theoretical background}
Particle Swarm Optimization (PSO) is a population-based optimization algorithm that simulates the social behavior of birds or insects, such as flocking or swarming.
\subsection{Programming PSO, trial over a simple two-dimensional study case}
Usage of the Eggholder function
\section{Comparison to other generation methods}

\section{Conclusion}
\end{multicols}
\bibliographystyle{acm}
\bibliography{articles}

\end{document}