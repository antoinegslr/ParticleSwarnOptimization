\documentclass[11pt]{article}

% Language setting
% Replace `english' with e.g. `spanish' to change the document language
\usepackage[english]{babel}

% Set page size and margins
% Replace `letterpaper' with `a4paper' for UK/EU standard size
\usepackage[a4paper,top=2cm,bottom=2cm,left=2cm,right=2cm,marginparwidth=1.75cm]{geometry}

% Useful packages
\usepackage{amsmath,amssymb,amsfonts}
\usepackage{graphicx}
\usepackage[colorlinks=true, allcolors=blue]{hyperref}

\title{\textbf{Generation of optimized structures using Particle Swarn Optimization (PSO)}}
\author{Antoine GISSLER}
\date{January 10th, 2023}

\begin{document}
\maketitle

\section{Introduction}
For as long as both theoretical and technological progress entitled scientists to do, there has been a wide interest to understand the nanoscopic scale of matter. Indeed, doing so makes it possible to understand the processes and the results of changement inside the matter (phase transition) \\

This report, based on the publication from Robinson et al. \cite{original}, tackles the application of a generalized optimization algorithm (Particle Swarn Optimization) for the purposes of potential energy surfaces exploration and the determination of the optimized geometry for particular conditions.

%Les simulations de systèmes moléculaires, qu'ils soient complexes ou non, nécessitent l'utilisation de configurations initiales. 
%Faire une introduction sur la nécessité de devoir générer des configurations par la théorie dans le cas de conditions expérimentales exotiques (on ne peut pas se baser sur des expériences précédentes car il n'y en a pas lol)
\section{Particle Swarn Optimization}
Particle Swarm Optimization (PSO) is a population-based optimization algorithm that simulates the social behavior of birds or insects, such as flocking or swarming.
\section{Comparison to other generation methods}

\section{Conclusion}

\bibliographystyle{acm}
\bibliography{articles}

\end{document}